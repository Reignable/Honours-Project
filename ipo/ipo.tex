\documentclass[a4paper, 12pt]{article}

\title{Initial Project Overview}
\author{Joe Barrett}

\usepackage{fancyhdr}
\usepackage{fontspec}
\usepackage{hyperref}
\usepackage[ddmmyyyy]{datetime}
\usepackage[margin=1in]{geometry}
\usepackage[none]{hyphenat}

\setmainfont{Arial}

\pagestyle{fancy}
\renewcommand{\headrulewidth}{0pt}

\lhead{Joe Barrett}
\rhead{40117680}
\lfoot{BEng Software Engineering}
\cfoot{\today}
\rfoot{\thepage}

\begin{document}
	\setcounter{secnumdepth}{-1}
	\pagenumbering{arabic}
	\section[Heading]{\Large Initial Project Overview
			\newline
			SOC10101 Honours Project (40 Credits)
	}
	
	\subsection[Title]{Title of Project:}
	Calculation of Mountain Bike Suspension Setup through Mobile Image Processing
	
	\subsection[Overview]{\underline{Overview of Project Content and Milestones}}
	Due to the lack of knowledge and complexity regarding rear suspension setup on mountain bikes, many riders have an improper setup which can potentially lead to injury. The purpose of this project will be to design and produce a mobile application capable of providing a suggested rear suspension setup for the user based on images of the bike frame and user provided information.
	\newline\newline
	The project will consist of research into current image processing uses, techniques, and mobile applications which use image processing. Design and implementation of the prototype mobile application. Finally, an evaluation of the prototype application.
	
	\subsection[Deliverables]{The Main Deliverable(s):}
	\begin{itemize}
		\item A literature review of image processing techniques
		\item An analysis of currently available applications and products in the related area
		\item A prototype mobile application capable of suggesting a rear suspension setup for the user
		\item A critical evaluation of the prototype application against currently available applications
	\end{itemize}
	
	\subsection[Audience]{The Target Audience for the Deliverable:}
	Entry level to intermediate mountain bikers with little to no knowledge of the rear suspension setup process.
	
	\subsection[Work]{The Work to be Undertaken:}
	\begin{itemize}
		\item Investigation into image processing techniques, uses, and current applications
		\item Produce a literature review of image processing techniques
		\item Produce a design for the prototype mobile application
		\item Implement the design into a working prototype application
		\item Evaluate the prototype
	\end{itemize}
	
	\subsection[Additional Info]{Additional Information / Knowledge Required:}
	\begin{itemize}
		\item Knowledge of OpenCV library or alternative image processing libraries
		\item Improved knowledge of Android\texttrademark\ framework and programming
		\item Improved knowledge of mountain bike rear suspension setup
	\end{itemize}
	
	\subsection[Sources]{Information Sources that Provide a Context for the Project:}
	\begin{itemize}
		\item OpenCV Image Processing Library for Android\texttrademark, \href{www.opencv.org}{www.opencv.org}
		\item Introduction to Video and Image Processing Building Real Systems and Applications, Thomas B Moeslund, 2012
		\item SussMyBike Data Acquisition - Eurobike 2016,\newline
		\href{www.pinkbike.com/news/sussmybike-data-acquisition-eurobike-2016}{www.pinkbike.com/news/sussmybike-data-acquisition-eurobike-2016}
	\end{itemize}
	
	\subsection[Importance]{The Importance of the Project:}
	Correct suspension setup on a mountain bike improves its ride-ability and reduces excessive wear and tear on components leading to further enjoyment of the sport. After conversations with The Mountain Bike Research Centre of Scotland it has been shown that a drastically incorrect setup can cause the rider to crash which may lead to injury and can potentially be fatal. 
	
	\subsection[Challenges]{The Key Challenge(s) to be Overcome:}
	\begin{itemize}
		\item Personal understanding of image processing techniques
		\item Use of image processing to calculate valuable metrics
		\item Carrying out a critical and scientific evaluation of the prototype
	\end{itemize}
\end{document}

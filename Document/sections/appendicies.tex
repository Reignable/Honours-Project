\appendix
\includepdf[scale=0.9,
			pages=1,
			pagecommand=\section{Initial Project Overview},
			offset=0 -1cm]{../ipo/ipo.pdf}
\includepdf[scale=0.9,
			pages=2-,
			pagecommand={}]{../ipo/ipo.pdf}
\includepdf[scale=0.9,
			pages=1,
			pagecommand=\section{Week 9 Report}]{../week_9_report.pdf}
\includepdf[scale=0.9,
			pages=2-,
			pagecommand={}]{../week_9_report.pdf}
\clearpage
\section{Android Experiments}\label{app:android_experiments}
	\subsection{Table of Android Experiments}
	\rowcolors{2}{gray!25}{white}
	\begin{table}[h!]
		\centering
		\label{tab:android_experiments}
		\begin{tabular}{|l|p{0.3\textwidth}|l|p{0.2\textwidth}|r|r|}
			\hline
			\rowcolor{gray!50}
			\bfseries Experiment&\bfseries Purpose&\bfseries Works&\bfseries Issues&\bfseries Files&\bfseries LOC\\
			\hline
			Hello CV&
			Introduction to the android OpenCV library. Displays camera feed with fps.&
			\checkmark&
			\begin{itemize}[noitemsep,topsep=0pt,parsep=0pt]
				\item{Not fullscreen}
				\item{Incorrect orientation}
			\end{itemize}&
			2&
			101\\
			15Tile&
			Sliding tile game. Uses camera feed as puzzle.&
			\checkmark&
			\begin{itemize}[noitemsep,topsep=0pt,parsep=0pt]
				\item{Not fullscreen}
			\end{itemize}&
			3&
			492\\
			Blob Detection&
			Demonstrates blob detection. Runs blob detection on tapped area from camera.&
			X&
			\begin{itemize}[noitemsep,topsep=0pt,parsep=0pt]
				\item{Crash on screen tap}
			\end{itemize}&
			3&
			311\\
			Face Detection&
			Detects faces in camera view. Puts boundary around detected faces.&
			X&
			\begin{itemize}[noitemsep,topsep=0pt,parsep=0pt]
				\item{Crash on load}
			\end{itemize}&
			3&
			279\\
			\hline
		\end{tabular}
	\end{table}
\subsection{Hello CV}
	\inputminted[breaklines,
					linenos,
					frame=lines,
					fontsize=\footnotesize]{java}{../code/android/hello_cv/MainActivity.java}
	\subsection{15tile}
	\inputminted[breaklines,
					linenos,
					frame=lines,
					fontsize=\footnotesize]{java}{../code/android/15tile/MainActivity.java}
	\inputminted[breaklines,
					linenos,
					frame=lines,
					fontsize=\footnotesize]{java}{../code/android/15tile/PuzzleProcessor.java}
	\subsection{BLOB Analysis}
	\inputminted[breaklines,
					linenos,
					frame=lines,
					fontsize=\footnotesize]{java}{../code/android/blob_analysis/ColorBlobDetectionActivity.java}
	\inputminted[breaklines,
					linenos,
					frame=lines,
					fontsize=\footnotesize]{java}{../code/android/blob_analysis/ColorBlobDetector.java}
	\subsection{Face Detection}
	\inputminted[breaklines,
					linenos,
					frame=lines,
					fontsize=\footnotesize]{java}{../code/android/face_recognition/DetectionBasedTracker.java}
	\inputminted[breaklines,
					linenos,
					frame=lines,
					fontsize=\footnotesize]{java}{../code/android/face_recognition/FrActivity.java}
\clearpage
\section{Python Experiments}\label{app:python_experiments}
		\subsection{Table of Python Experiments}
		\begin{table}[h!]
			\centering
			\label{tab:python_experiments}
			\begin{tabular}{|l|p{0.4\textwidth}|l|l|r|r|}
				\hline
				\rowcolor{gray!50}
				\bfseries Experiment&\bfseries Purpose&\bfseries Works&\bfseries Issues&\bfseries Files&\bfseries LOC\\
				\hline
				Find Game&
				Identify red game cartridge out of 3. Displays boundary around correct part of image.&
				\checkmark&
				N/A&
				1&
				18\\
				Threshold Methods&
				Demonstrates various thresholding methods on an image. Original image text unreadable but clear after techniques are applied.&
				\checkmark&
				N/A&
				1&
				14\\
				Image Operations&
				Moves parts of an image to other locations using Arrays. Demonstrates how pixel data is stored.&
				\checkmark&
				N/A&
				1&
				15\\
				Distance to Camera&
				Calculates distance to the camera from an identified object. Displays various distances using 3 images.&
				\checkmark&
				N/A&
				1&
				38\\
				\hline
			\end{tabular}
		\end{table}
		\subsection{find\_game.py}
		\inputminted[breaklines,
						linenos,
						frame=lines,
						fontsize=\footnotesize]{python}{../code/python/find_game.py}
		\subsection{thresholding.py}
		\inputminted[breaklines,
						linenos,
						frame=lines,
						fontsize=\footnotesize]{python}{../code/python/thresholding.py}
		\subsection{img\_ops.py}
		\inputminted[breaklines,
						linenos,
						frame=lines,
						fontsize=\footnotesize]{python}{../code/python/img_ops.py}
		\subsection{distance\_to\_camera.py}
		\inputminted[breaklines,
						linenos,
						frame=lines,
						fontsize=\footnotesize]{python}{../code/python/distance_to_camera.py}
\clearpage
\section{EXIF Extraction Code}\label{app:exif_code}
	\inputminted[breaklines,
				linenos,
				frame=lines,
				fontsize=\footnotesize,
				firstline=36,
				lastline=63]{python}{../code/program/v2.py}
\clearpage
\includepdf[scale=0.8,
pages=1,
pagecommand=\section{Development Log}\label{app:dev_log}]
{../dev_log.pdf}
\includepdf[scale=0.8,
pages=2-]{../dev_log.pdf}
\clearpage
\section{Git Commit Log}\label{app:commit_log}
\inputminted[breaklines=true]{text}{../git_log.txt}
\clearpage
\section{Gantt Charts}\label{app:gantts}
	\subsection{Original}
		\begin{figure}[h!]
			\includegraphics[width=\textwidth]{../images/gantt/orig/gantt.PNG}
		\end{figure}
		\begin{figure}[h!]
			\includegraphics[]{../images/gantt/orig/tasks.PNG}
		\end{figure}
	\clearpage
	\subsection{First Revision}
		\begin{figure}[h!]
			\includegraphics[width=\textwidth]{../images/gantt/v1/gantt.PNG}
		\end{figure}
		\begin{figure}[h!]
			\includegraphics[]{../images/gantt/v1/tasks.PNG}
		\end{figure}
	\clearpage
	\subsection{Second Revision}
		\begin{figure}[h!]
			\includegraphics[width=\textwidth]{../images/gantt/v2/gantt.PNG}
		\end{figure}
		\begin{figure}[h!]
			\includegraphics[]{../images/gantt/v2/tasks.PNG}
		\end{figure}
	\clearpage
	\subsection{Third Revision}
		\begin{figure}[h!]
			\includegraphics[width=\textwidth]{../images/gantt/v3/gantt.PNG}
		\end{figure}
		\begin{figure}[h!]
			\includegraphics[]{../images/gantt/v3/tasks.PNG}
		\end{figure}
\clearpage
\section{Statements From Evaluation Meeting}\label{app:eval_statements}
\subsection{Geraint Florida-James}
"I have spent some time with Joe looking at his work and there is definitely a useful application in development. This as expected is not a full resource to deal with the vast number of variables that constitute mountain bike suspension set up but this application helps the user on the very first, and most important, step of this process. With the evidence I have seen I would encourage further development of the next steps. "
\clearpage
\section{Source Code}\label{app:source}
	\subsection{main.py}
		\inputminted[breaklines,
					linenos,
					frame=lines,
					fontsize=\footnotesize]{python}{../code/final_program/main.py}
	\subsection{image\_processor.py}
		\inputminted[breaklines,
					linenos,
					frame=lines,
					fontsize=\footnotesize]{python}{../code/final_program/image_processor.py}
	\subsection{pressure\_calculator.py}\label{app:source_pressure_calculator}
		\inputminted[breaklines,
					linenos,
					frame=lines,
					fontsize=\footnotesize]{python}{../code/final_program/pressure_calculator.py}
	\subsection{test\_image\_processor.py}
		\inputminted[breaklines,
					linenos,
					frame=lines,
					fontsize=\footnotesize]{python}{../code/final_program/test_image_processor.py}
	\subsection{test\_pressure\_calculator.py}
		\inputminted[breaklines,
					linenos,
					frame=lines,
					fontsize=\footnotesize]{python}{../code/final_program/test_pressure_calculator.py}
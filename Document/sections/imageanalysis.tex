\subsection{Image Analysis}
	\gls{ia} is the use of various techniques such as pattern recognition, geometry calculations, and signal processing to extract information from digital images for later use. Image processing however is the application of various processes on an image to change or improve the way it looks. The processing stage normally comes before the analysis stage in an effort to simplify the analysis processes and improve their success.
	\subsubsection{Usages}
	Image processing and analysis has been applied to multiple areas with its value and effectiveness rapidly improving alongside camera technology and computing power. These applications range from recognising faces in social media uploads \citep{zuckerberg2011tagging} to the utilisation of satellite imagery to tracking the changing shape of coastlines \citep{costalimagery}.
	\paragraph{Medical}
	Arguably one of the most important uses of \gls{ia}, advances in medical imaging have reduced costs in healthcare, diagnosis time, recovery time, and improved the ability to localise and personalise treatments \citep{esfmedical}. Major uses of \gls{ia} in medical applications are the use of Magnetic Resonance Imaging (MRI) and Computerised Topography Scanning (CT Scan) to create detailed images of the human body and identify illness before some symptoms arise.
	\begin{comment}
	\begin{figure}[h!]
		\centering
		\includegraphics[width=10cm]{../images/mri.jpg}
		\caption{Identification of arthritis in an MRI scan \citep{mriimage}}			
		\label{fig:mri}
	\end{figure}
	\end{comment}
	\paragraph{Transport}
	Image analysis has been included in the consumer automotive market on various models since 2004 when Honda introduced an thermographic night vision camera with pedestrian detection on the Legend  \citep{hondanightvision}. Since this initial introduction many vehicle manufacturers have included image analysis and recognition features as options such as speed limit sign recognition, lane departure warning systems, and automatic braking systems based on hazard recognition.
	\paragraph{Engineering}
	The use of image analysis in engineering has pushed to create more stable and efficient structures by looking at the materials used in their construction \citep{concreteanalysis} and monitoring their stresses and potential weak areas \citep{bridgecables}. Using image analysis by engineers on site has become more common with the advances in mobile computing and some manufacturers aiming their products at an engineering demographic with features like improved durability and built in infrared imaging \citep{catphone}.
	\paragraph{Space}
	While some industries make use of satellite imagery to monitor changes on our own planet, agencies such as NASA and ESA make use of image analysis to look at planets and other celestial bodies. The Martian rover, Curiosity, uses multiple cameras for navigation, hazard avoidance, and scientific imaging the products of which are streamed back to Earth for analysis. Major uses for the various types of images returned include identification of geological formations and compositions \citep{curiositysand, curiositygravel} and chemical location and identification using the "ChemCam" \citep{curiosityhydrogen}.
	\begin{comment}
	\begin{figure}[h!]
		\centering
		\includegraphics[width=7cm]{../images/curiosity.jpg}
		\caption{The Martian rover, Curiosity}			
		\label{fig:curiosity}
	\end{figure}
	\end{comment}
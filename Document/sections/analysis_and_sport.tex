\subsection{Image Analysis in Sports Science}
	The benefits of image analysis in sports science come to light when used in biomechanical situations. In most cases, an individual's performance in a sport depends on their physical fitness and technique; including the controversial topic of driver-athletes in motorsport \todo{Cite something}. Despite being surrounded by the engineering of their vehicle or equipment, if the driver is unable to cope with the stresses and strains of race conditions then they will not succeed. 
	\\\\
	A car, motorcycle, and mountain bike are all predictable and measurable as they have been designed and manufactured by engineers so can easily be fitted with sensors, analysed, and adjusted accordingly \todo{Cite}. Fitting sensors to humans while maintaining their full mobility is not so simple. To provide the best testing ground for fitness and technique, the participant should be unhindered and able to perform tasks without data capturing equipment getting in the way.
	\\\\
	Image analysis can play a large part in this as utilising various techniques can allow for stable and repeatable test situations while also providing a platform for review in the captured images. A study into cricket bowling technique used a mixture of manual point picking and automated measuring from images to produce data such as angle and speed of bowling deliveries as well as ball spin \citep{cricketimaging}. While the dataset was limited due to the player's training schedules, the data collected was useful as it was then used for replication on a pitching machine.
	\\\\
	A second study made use of similar techniques to Cook, Justham, and West but these were applied to training horses for jumping competitions \citep{jumpyhorses}. The angles of limbs were recorded and compared over a period of four months to analyse improvements made from training. Using passive image analysis for this study would have been preferable over attaching sensors to the horse as this could have frightened the animal and almost certainly cause it to not have performed effectively.
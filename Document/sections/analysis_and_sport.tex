\subsection{Image Analysis in Sports Science}
	The benefits of image analysis in sports science are most prevalent in the area of biomechanics. In most sports , an individual’s performance depends on a combination of physical fitness and acquired skills. In certain sports, such as motorsports, equipment is undeniably important although unless “driver-athletes” can cope with the stresses and strains of race conditions they are unlikely to achieve success regardless of the technical superiority of their vehicle \citep{klarica2001performance}.
	\\\\
	In such sports, the performance of cars, motorcycles, and even mountain bikes are all predictable and measurable. They have all been designed and manufactured to be as fast as possible within the rules laid down by the sport’s governing body so it is important to continually analyse and fine tune their set – up by changing tires or adjusting suspension to seeking out the margin gains that make the difference between gaining a place on the podium or not. As a result, performance cars and bikes are fitted with an array of sensors that constantly monitor critical aspects of performance such as engine temperatures or suspension \citep{segers2008analysis}. 
	\\\\
	Fitting sensors to humans without impairing their mobility is not quite so simple. To provide the best testing ground for fitness and technique, the participant should be unhindered and able to perform tasks without data capturing equipment getting in their way. Image analysis get around such issues by utilising various techniques that can allow for stable and repeatable test situations while also providing a platform for reviewing the captured images. A study into bowling techniques in cricket used a mix of manual point picking and automated measuring from images to produce data such as the angle and speed of bowling deliveries as well as ball spin \citep{cricketimaging}. While the dataset was limited due to conflicts with the players’ training schedules, the results proved useful and were subsequently replicated in a pitching machine so that batsmen practised against more lifelike deliveries.
	\\\\
	A second study on the use of IA in showjumping \citep{jumpyhorses} made use of techniques similar to those used by Cook, Justham, and West. In this instance, the angles of the horse’s limbs were recorded and compared over a period of four months to analyse whether different training techniques delivered measurable improvements. Here, passive image analysis was chosen as the preferred technique as attaching sensors to the horse could have frightened the animal and almost certainly caused it perform below par.
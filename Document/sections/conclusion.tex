\section{Conclusions}\label{sec:conclusion}
	This section will conclude the project by evaluating how well the original aims have been met to determine the success of the project, compare the prototype application to current products on the market, and outline any future work which can be potentially completed in subsequent projects. Finally a self appraisal will be carried out to discuss this author's performance during the project.
	\subsection{Meeting Aims}
		% How well does the solution relate to the original aims and objectives
			% Very well
		As a measure of success the original project aims which, were explained in section \ref{sec:aims_and_objectives}, can be examined to see if they have been met upon project completion. 
		\subsubsection{Aim 1}
			The first aim was to complete a literature review of mountain bike suspension and image analysis techniques including how image analysis is currently used in sports science which has been met by section \ref{sec:lit_review}. This literature review presented the fundamentals of mountain bike suspension to provide insight into its operation as a way of building knowledge of the project context. The research into image analysis uses and techniques proved useful in deciding how the application would operate which aided the development process, particularly when issues appeared with the measurement technique and reference point finding methods.
		\subsubsection{Aim 2}
			The second aim was to implement the prototype application using identified and researched methods. This aim has been met using methods selected from those outlined in section \ref{sec:methodology} and the resulting application is documented in section \ref{sec:results}.
			\\\\
			% Dev process
			The development approach chosen allowed for the project to run smoothly and produce the application on time. Though an agile approach may have been able to identify potential issues earlier in the development stage, using design and analysis techniques, it was felt that this would have put unnecessary pressure on the project in the form of excessive documentation and structure. The lose organic approach taken let issues be resolved as they arose to maintain momentum in the process. 
			\\\\
			This process also allowed the project to continue despite not knowing the exact algorithmic approach the application would take. It was felt that, with the time available, it was best to begin development as soon as possible and allow the project to adapt accordingly as time went on rather than rigorously structure the process. This lead to a change in the image analysis techniques used for reference point finding and a complete change from trying to create a pressure per millimetre metric to utilising two images and the linear equation. It is believed that if the project had included extensive analysis and design of the application then the creative freedom would have been reduced and these solutions may not have been discovered. For larger projects with real stakeholders this would not be suitable but in this experimental context the approach worked well.
			\\\\
			% Other PM, git, gantt
			Using the Git version control system outlined in section \ref{sec:methodology_version_control} allowed for the project's documents and source code to be easily backed up and tracked. Indicated by the commit logs shown in appendix \ref{app:commit_log}, the system was well used with regular commits and numerous branches. The Gantt charts, described in section \ref{sec:methodology_gantt} and shown in appendix \ref{app:gantts}, were regularly checked and updated throughout the project to track each stage and its completion. Multiple versions were produced as the chart was updated whenever the timeline was significantly adjusted, this included putting the project on hold for a period of 2 weeks and moving the deadline forward.
			\\\\
			% Platform, testing
			By producing a Python application as opposed to an Android mobile application, the project was able to produce a more capable product than it would have on the Android platform. Time was saved by not having to produce a user interface and handle the difficulties of using OpenCV on Android which let the image analysis and calculations become more advanced. By implementing unit tests it could be confirmed that the application was functioning correctly throughout development without having to manually test each time. A push could have been made for 100\% test coverage but as described in section \ref{sec:evaluation_validation}, this was not vital.
		\subsubsection{Aim 3}
			% Evaluation of different metrics completed
			% Proves the application works as expected, produces good results, and is well received
			The third aim was to evaluate the success and appropriateness of the produced application, this has been completed and is documented in section \ref{sec:evaluation}. By evaluating a selection of metrics it has been proven that the application functions as expected, produces near suitable results, and is well received by industry professionals. Additionally this evaluation process has shown that the project was capable of producing the original concept for the application and has created a basis for future work.
		\subsubsection{Aim 4}
			% Done in this section
			The final aim was to present conclusions about the project's successes and downfalls which is shown in the present section.
	\subsection{Comparison to similar Products}
		% Fox is locked to their product, requires lots of alignment
		% Shockwiz, needs to be ridden and attached, expensive
		Due to the price and availability of the Shockwiz device a unit was not available to this project for comparison. Additionally as the Fox IRD application is only found on Apple's iOS, it could not be experimented with. However a comparison can still be made by using information about the two products as well as user's experiences.
		\subsubsection{Shockwiz}
			% Aimed at riders wanting to tune, rather than setup
			% Expensive, needs additional device
			The Shockwiz data logging device \citep{quarq2017shockwiz} is capable of analysing characteristics about front and rear air suspension while it is being ridden and suggesting adjustments to the user through a mobile application. In comparison to the application produced during this project it provides rebound and compression settings in conjunction with the basic air pressure. Each adjustment works on a sliding scale as opposed to set numbers which is helpful when tuning suspension.
			\\\\
			Due to the number of settings and level of detail provided to the user, alongside the £359 price, it is clear that the Shockwiz system is aimed at intermediate to expert riders wanting to tune their suspension for various trails. The target audience for this project is beginner riders wanting to produce a baseline setup with minimal effort. 
		\subsubsection{Fox IRD}
			The Fox Intelligent Ride Dynamics application \cite{fox2015ird} is capable of using a smartphone's camera to produce a sag setting for a fork or shock. This is carried out in a similar way to the application from this project. Additionally the IRD application can suggest a rebound setting dynamically, it is unsure how this is produced.
			\\\\
			While this project's application has been tested on shocks from two manufacturers, the Fox's application is tied to their own suspension units. A further caveat is that it's tied to the 2013 model year. This means that the application cannot be used with older and newer suspension which has resulted in complaints from customers. The locking of the application to one manufacturer and model year suggests that the application was created as an experiment rather than a viable product.
	\subsection{Future Work}\label{sec:conclusion_future_work}
		\subsubsection{Mobile Application}
		% Make into app
			% Python wrapper
			% Ref back to experiments on difficulties
			A clear section of work would be to create a smartphone application based around the one produced in this project. This would move the application closer to the original vision of this project and allow for a simpler user experience buy using the smartphone to produce the images and complete integrated analysis.
			\\\\
			This could be completed by including the source code from this application by means of a wrapper \citep{kivy2015python} or scripting layer \citep{asl}. However issues are presented by both of these methods. Development by the original team on the Python-for-Android wrapper has stopped which means and remaining bugs with the system will not be fixed. This is also the case with the Android Scripting Layer which was also never taken further than an alpha level application so it is likely to be unstable and incapable of the required functionality.
			\\\\
			An alternative method would be to use the OpenCV package for Android and recreate the algorithms from this project natively though, as previously described in section \ref{sec:methodology_platform_experiments}, this also presents difficulties. With enough time and resources available the issues encountered in this project may be rectified with a correctly setup and configured environment. Alternatively, the other image analysis libraries available for Android could be considered.
		\subsubsection{Expanded Functionality}
			% Add functionality (Specify pressures, different units)
			% Make work for all shocks
			% Make work for front suspension
			% Add rebound and compression (database)
			% Coil
			Additional work would also be to expand the functionality of the application. Currently it can process images of rear shocks but this should be expanded to include the fork, allowing users to set up both suspension units. This would be a case of adapting the current image analysis to work on the different images. The application has also been tested on two varieties of shock, this should be expanded to other manufacturers and models in a variety of situations and adapted accordingly to assure its capabilities.
			\\\\
			Secondly the application should be adapted to work with coil shocks as opposed to working with only air suspension. This could be done by locating the bolts which hold the shock into the frame and measuring the distance between them, which could be simple as they are circular though commonly have other components blocking them from view. A further difficulty arises as there are no marker O-rings on coil shocks so the rider would have to be seated on the bike when the image is taken.
			\\\\
			As discussed in section \ref{sec:alt_methods}, the disparity between the manually produced and computationally produced pressures should be resolved. To do so the application should be altered to allow the user to specify which pressures the images are taken at, as opposed to being locked to 100PSI and 150PSI. This would make the application adaptable for higher volume shocks by supplying pressures which encompass the target pressure, 150PSI to 250PSI for example.
			\\\\
			Finally the application could be expanded to suggest compression and rebound damping settings alongside the sag setting. This would allow the user to setup every aspect of their suspension before riding for the first time. One method would be to dynamically predict a suitable setting using the produced sag setting, intended riding style, and current suspension design. Alternatively, an online database of suggested settings could be collated and queried with the user's data. This database could be included as part of a wider scoped project including a website allowing users to look up settings from those contributed by other users rather than using the application.
	\subsection{Self Appraisal}
		This project was built from my previous experience in the cycling industry as a mechanic and from my learnings from a previous project to produce a device similar to Shockwiz. This knowledge aided addressing the problem from a mountain biking context though throughout my academic career I had not developed an application which utilises image analysis. By using research and experimentation into suitable techniques and platforms the proposed application was implemented successfully.
		\\\\
		As this is the largest project I have completed to date, in terms of timeline and amount of research required, it was an unknown as to whether I would be able to complete it to a standard which I would deem suitable. Many factors have interfered with the amount of time available to work on this project such as other university commitments and taking time out to attend job interviews. Despite this, by keeping track of the stages of the project and remaining work in both a mental manner and in documentation, the project remained on track and constantly kept in a suitable position relative to the time remaining. Following the time in which the project was placed on hold I found it difficult to resume work as the development stage was due to begin. Thanks to note taking and planning carried out beforehand the process of resuming the project had been made simpler and work was able to continue with good momentum.
		\\\\
		Midway through the development stage the Git commit logs were consulted to trace the source of an issue in the application. While doing this it was found that the commit messages were not descriptive of the work which had been completed which made locating the issue difficult and time consuming. If it were an individual other than myself carrying out this investigation, the lack of description would have been a serious issue. Subsequent commit messages were made more descriptive should the need to consult them ever arise again. This can be placed in conjunction with the level of documentation kept throughout the project. Though it may have been suitable for this size of project and did not affect its success, in a larger context this level of planning and documentation may not be suitable and will be taken into account in the future.
		\\\\
		Overall I believe I have performed well in completing this project, which is indicated by the suitability of the application as a solution to the problem identified. It is gratifying to have seen my leanings and abilities put into practice as well as gaining new knowledge and methods from overcoming the issues faced throughout the course of the project.
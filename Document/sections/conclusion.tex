\section{Conclusions}\label{sec:conclusion}
	This section will conclude the project by evaluating how well the original aims have been met to determine the success of the project, compare the prototype application to current products on the market, and outline any future work which can be potentially completed in subsequent projects. Finally a self appraisal will be carried out to discuss this author's performance during the project.
	\subsection{Meeting Aims}
		% How well does the solution relate to the original aims and objectives
			% Very well
		As a measure of success the original project aims which were explained in section \ref{sec:aims_and_objectives}, can be examined to see if they have been met now the project is complete. 
		\subsubsection{Aim 1: Literature Review}
			The first aim was to complete a literature review of mountain bike suspension and image analysis techniques including how image analysis is currently used in sports science. This has been met in section \ref{sec:lit_review}. This literature review presented the fundamentals of mountain bike suspension to provide insight into its operation as a way of building knowledge of the project context. Research into the various techniques of image analysis and their uses proved useful in deciding how the application would operate. This aided the development process, particularly when issues were encountered in finding the reference point and subsequently in gaining accurate measurements.
		\subsubsection{Aim 2: Prototype Application}
			The second aim was to implement the prototype application using methods identified and researched as part of the literature review. This aim has been met using methods selected from those outlined in section \ref{sec:methodology} with the resulting application is documented in section \ref{sec:results}.
			\\\\
			% Dev process
			The development approach chosen allowed for the project to run smoothly and produce the application on time. Although adopting and agile approach may have resulted in the earlier identification issues encountered during development by incorporating ongoing design and analysis techniques, it was felt that this would have put unnecessary pressure on the project in the form of excessive documentation and structure. The looser organic approach taken allowed for issues to be resolved as they arose to maintain momentum in the process. 
			\\\\
			Adopting an organic approach also allowed the project to continue despite not knowing the exact algorithmic approach the application would take. Given the limited time available it was felt that it was best to begin development as soon as possible and allow the project to adapt accordingly as time went on rather than rigorously structure the process. This lead to a change in the image analysis techniques used for reference point finding and a complete change from trying to create a pressure per millimetre metric to utilising two images and the linear equation. It is believed had creative freedom been curtailed by included extensive analysis and exhaustive design into the process, these solutions may have not been discovered. For larger projects with multiple stakeholders this would not be suitable but in this experimental context the approach worked well.
			\\\\
			% Other PM, git, gantt
<<<<<<< HEAD
			Using the Git version control system outlined in section \ref{sec:methodology_version_control} allowed for the project's documents and source code to be easily backed up and tracked. The commit logs shown in appendix \ref{app:commit_log} demonstrate that the system was well used with regular commits and numerous branches. The Gantt charting described in section \ref{sec:methodology_gantt} were regularly checked and updated throughout the project to track each stage and its completion. Multiple versions were produced as the chart was updated whenever the timeline was significantly adjusted. This included putting the project on hold for a period of 2 weeks and moving the deadline forward.
=======
			Using the Git version control system outlined in section \ref{sec:methodology_version_control} allowed for the project's documents and source code to be easily backed up and tracked. Indicated by the commit logs shown in appendix \ref{app:commit_log}, the system was well used with regular commits and numerous branches. The Gantt charts, described in section \ref{sec:methodology_gantt} and shown in appendix \ref{app:gantts}, were regularly checked and updated throughout the project to track each stage and its completion. Multiple versions were produced as the chart was updated whenever the timeline was significantly adjusted, this included putting the project on hold for a period of 2 weeks and moving the deadline forward.
>>>>>>> 72874faeca38e7f1aa46081642a30d36b2be46e8
			\\\\
			% Platform, testing
			By producing a Python application as opposed to an Android mobile application, the project was able to produce a product with greater capabilities than it would have on the Android platform. Not having to produce a user interface and handle the difficulties of using OpenCV on Android allowed more time to research and develop more advanced image analysis techniques and greater sophistication in the underlying calculation. By implementing unit tests it could be confirmed that the application was functioning correctly throughout development. A push could have been made for 100\% test coverage but as described in section \ref{sec:evaluation_validation}, this was not vital.
		\subsubsection{Aim 3: Evaluation}
			% Evaluation of different metrics completed
			% Proves the application works as expected, produces good results, and is well received
			The third aim was to evaluate the success and appropriateness of the produced application, this has been completed and is documented in section \ref{sec:evaluation}. By evaluating a selection of metrics it has been proven that the application functions as expected, produces reliable results, and is well received by industry professionals. Additionally this evaluation process has shown that the project was capable of producing the original concept for the application and has created a basis for future work.
		\subsubsection{Aim 4: Conclusions}
			% Done in this section
			The final aim was to present conclusions about the project's successes and shortcomings which is shown in the present section.
	\subsection{Comparison to similar Products}
		% Fox is locked to their product, requires lots of alignment
		% Shockwiz, needs to be ridden and attached, expensive
		The high price and limited availability of the Shockwiz device a unit was not available for comparison. Additionally as the Fox IRD application is only offered on Apple's iOS, it was not available for testing and comparison. However a comparison can still be made by using anecdotal experience from users collected from various online discussion groups.
		\subsubsection{Shockwiz}
			% Aimed at riders wanting to tune, rather than setup
			% Expensive, needs additional device
			The Shockwiz data logging device \citep{quarq2017shockwiz} collects and analyses data about front and rear suspension units while the bike is being ridden and provides users with recommended adjustments through a mobile application. In comparison to the application produced during this project it provides rebound and compression settings in conjunction with the basic air pressure. Each adjustment works on a sliding scale as opposed to set numbers which is helpful when tuning suspension.
			\\\\
			Due to the number of settings and level of detail provided to the user, alongside the £359 price, it is clear that the Shockwiz system is aimed at intermediate to expert riders wanting to tune their suspension for various trails. The target audience for this project is beginner riders wanting to produce a baseline setup with minimal effort. 
		\subsubsection{Fox IRD}
			The Fox Intelligent Ride Dynamics application \cite{fox2015ird} uses a smartphone's camera to produce a sag setting for a fork or shock. This is carried out in a similar way to the application delivered by this project. Additionally the IRD application can suggest a rebound setting dynamically, although there is no documentation available as to how this is produced.
			\\\\
			While this project's application has been tested on shocks from two manufacturers, the Fox's application is tied to their own suspension units. A further caveat is that it is restricted to their 2013 model year. This means that the application cannot be used with either older and newer suspension which has resulted in complaints from customers. The locking of the application to one manufacturer and model year suggests that the application was created as an experiment rather than a viable product.
	\subsection{Future Work}\label{sec:conclusion_future_work}
		\subsubsection{Mobile Application}
		% Make into app
			% Python wrapper
			% Ref back to experiments on difficulties
			A clear section of work would be to create a smartphone application based around the one produced in this project. This would move the application closer to the original vision of this project and allow for a simpler user experience by using the smartphone to capture the images, complete the analysis, and provide the feedback.
			\\\\
			This could be completed by including the source code from this application by means of a wrapper \citep{kivy2015python} or scripting layer \citep{asl}. However there are issues with both of these methods. Development by the original team on the Python-for-Android wrapper has stopped which means remaining bugs with the system will not be fixed. This is also the case with the Android Scripting Layer which was also never taken further than an alpha level application so it is likely to be unstable and incapable of providing the required functionality.
			\\\\
			An alternative method would be to use the OpenCV package for Android and recreate the algorithms from this project natively though, as previously described in section \ref{sec:methodology_platform_experiments}, this also presents difficulties. With enough time and resources available the issues encountered in this project may be rectified with a correctly setup and configured environment. Alternatively, the other image analysis libraries available for Android could be considered.
		\subsubsection{Expanded Functionality}
			% Add functionality (Specify pressures, different units)
			% Make work for all shocks
			% Make work for front suspension
			% Add rebound and compression (database)
			% Coil
			The functionality of the application could also be expanded in line with the feedback provided by the industry experts. Currently it can process images of rear shocks but this should be expanded to include the fork, allowing users to set up both suspension units. This would be a case of adapting the current image analysis to work on the different images. The application has also been tested on two varieties of shock, this should be expanded to other manufacturers and models in a variety of situations and adapted accordingly to assure its capabilities.
			\\\\
			Secondly the application should be adapted to work with coil shocks as opposed to just air suspension. This could be done by locating the bolts which hold the shock into the frame and measuring the distance between them, which could be simple as they are circular. However commonly concealed from view by other components. A further difficulty arises as there are no marker O-rings on coil shocks so the rider would have to be seated on the bike when the image is taken.
			\\\\
			As discussed in section \ref{sec:alt_methods}, the disparity between the manually produced and computationally produced pressures should be resolved. To do so the application should be altered to allow the user to specify which pressures the images are taken at, as opposed to being locked to 100PSI and 150PSI. This would make the application adaptable for higher volume shocks where pressures can be as high as 350PSI.
			\\\\
			Finally the application could be expanded to suggest compression and rebound damping settings alongside the current sag setting. This would allow the user to setup every aspect of their suspension before riding for the first time. One method would be to dynamically predict a suitable setting using the produced sag setting, intended riding style, and current suspension design. Alternatively, an online database of suggested settings could be collated and queried with the user's data. This database could be included as part of a wider scoped project including a website allowing users to look up settings from those contributed by other users rather than just using the application.
	\subsection{Self Appraisal}
		This has been the largest project I have ever undertaken in terms of timeline and amount of research required and at the outset I was unsure as to whether I could  complete it to the required standard. Many factors have interfered with the amount of time available to work on this project such as other university commitments and repeatedly taking time out to attend job interviews. Such interruptions made it difficult to maintain focus particularly once the development stage started. However keeping a mental check on progress and routinely using documentation and project management tools to constantly juggle priorities and the amount of time available meant I was always able to maintain good momentum.
		\\\\
		Midway through the development stage, I had to refer to the Git commit logs to trace the source of an issue in the application. While doing this, I found that the commit messages were not sufficiently descriptive of the work that had already been completed which made
		it difficult and time consuming to locate the issues. Had it been someone other than myself carrying out this investigation, this lack of description would have been a serious issue. Subsequently I made commit messages considerably more descriptive so the detail was always available should I need to refer to them again. The same issues apply to the level of detail recorded in project documentation which although suitable for a small, solo project would not be appropriate in larger projects involving more dependencies and a larger number of stakeholders. 
		\\\\
		Overall I believe I have performed well in completing this project and successfully delivering an application that meets its requirements. Constant interruptions and a string of issues, such as the initially inability to detect the reference point, meant time was sometimes scarce even when it was most needed in researching and testing new approaches to problems. However, in retrospect, I am gratified to have been able to apply the methods and techniques I have acquired during my course to overcome the many issues encountered during this project.
		
\subsection{Introduction}
	This chapter will give an outline of the methodologies used to complete the work identified in the previous chapters as well as the reasons for using these methods and alternatives which could have been used. The effectiveness of these methods can determine the success of the project so the chosen methods were required to reliable and manageable. A technical approach was identified for the most appropriate way to produce a solution to the problem identified and a project management approach was decided on so that the project could remain on track and meet the required deadline.
\subsection{Platform}
	As a proof of concept, the final product of this software project could take a variety of forms. Inline with current products on the market an application for Android could be produced which would demonstrate the capabilities of image analysis on a mobile device. Alternatively, the image analysis algorithm can be produced in the Python programming language as a script.
	\subsubsection{Experimentation}
		To further understand what may be required to produce both the Android and Python based approaches some basic experiments were carried out to compare the two and aid in deciding which method to use. For the Android experiments some example applications which are bundled with the OpenCV library were hand copied and run on a mobile device. This allowed their output and functionality to be seen as well as experience in using OpenCV on Android. For the Python based approach, tutorials from the site www.pyimagesearch.com created by Adrian Rosebrock were followed and run on a desktop computer.
		\\\\
		Appendix \ref{app:android_experiments} shows the experiments which were carried out on the android platform. It should be noted that, while all applications do not show compilation errors and were adjusted to work with the updated version of Android which the device was using, only two of the four experiments work successfully. Although a stack trace of the errors produced they are not descriptive about the cause of the error. This is due to how OpenCV works on the Android system.
		\\\\
		Alongside the Android Software Development Kit (SDK), Google provide a Native Development Kit (NDK) to allow modules which were not written in Java to be run on Android devices. The NDK understands various programming languages and applied a Java wrapper around them which is capable of extracting their functionality; this must be applied when using OpenCV on Android as it is written in C++. An artefact of using the NDK is that it cannot convert the stack trace produced by errors in the C++ module and after research into this issue it was found that it is difficult to rectify.
		\\\\
		The two working Android experiments do not operate as expected either. Shown in appendix \ref{app:android_experiments} the issues with each can be seen. Although steps were taken to rectify the orientation in the Hello CV experiment and the full-screen issue in both, any fix which was applied was not accepted by the system.
\subsection{Project Management}
	\subsubsection{Agile Development}
		Introduced in 2001 with the writing of the Agile manifesto \citep{beck2001manifesto}, agile development methodologies focus on high quality software products over the rigorous design based structure of traditional methods. By utilising various techniques such as stand-up meetings, a strong customer focus, and sprint cycles, agile has become widely adopted in industry and has been proven to produce successful projects \citep{state_of_agile_2015}.
		\\\\
		There are multiple agile methodologies (XP, Scrum, DSDM) all with their own principles and techniques however it is commonplace for a company to create their own development method which picks items from each to tailor to their needs. As this is a solo project with no customer then one set methodology will not be used, instead a variety of methods which will aid the management of the project and increase the efficiency of the development process have been selected. These will be outlined in the following sections.
		\paragraph{Requirements Analysis}
		\paragraph{MoSCoW}
			First used in the DSDM agile framework, MoSCoW analysis or prioritisation is the process of taking the requirements of a software product and placing them into one of four categories; must, should, could, and won't have. These deliverables may be prioritised for the entire project or for individual sprint cycles depending on the size and manageability of the project.
			\\\\
			MoSCoW was created to provide customers a better understanding of software requirements. Opposed to using high, medium, and low priorities MoSCoW is more descriptive of what the prioritisation means for the software project. From a development point of view the prioritisation process allows developers to focus on the core requirements of the project first, creating a viable software solution early on with extra features being added later if the resources are available.
			\\\\
			This project will use MoSCoW to prioritise the requirements identified using the requirements analysis technique allowing the development process to complete the requirements in the correct order. This will ensure that the core aspects of the solution are implemented first creating a successful project early and allowing it to improve as time allows.
		\paragraph{Sprint Cycles}
			Used by Scrum development teams, a sprint is a timeboxed effort of work scheduled to take between one week and one month. At the start of a sprint goals are chosen from the project requirements which are to be completed by the end of the sprint. When a sprint is complete a retrospective is carried out by the development team to discuss what went well, what didn't, and how this can be rectified for the next sprint.
			\\\\
			This project will use sprints in the same manner as an agile development team as this will allow easier completion of requirements and management of the development process. Using sprints will ensure the time allotted for development is used effectively which will lead to a higher quality product at the end of the project.
	\subsubsection{Version Control}
		Large software projects produce multiple files of code and documentation which are vital and must be kept safe. Were the files to be lost then the project would be delayed or drawn to a close as the time and resources may not be available to recreate the lost data. To combat this any data relating to the project must be backed up, preferably on a cloud based system, to avoid loss and allow the project to continue should anything happen to the local copy of the data.
		\\\\
		For this project the Git \gls{vcs} will be used. Git provides free use of a cloud based repository to store any files relating to a project and allows for work to be carried out locally by cloning the repository on a computer. However \glspl{vcs} also enable the management of the previous versions of files including information such as the individual changes made, when those changes were made, and who made the changes. This is a powerful tool as it means the various sections of the project can be experimented on without the risk of damaging the project; if a change is unsuccessful then the repository can be reverted to a functional point.
		\\\\
		The web service used to host the repository will be github.com. This site was chosen as it provides unlimited free repositories as well as simple repository management tools. The website also provides issue tracking functionality; though they are not issues, each feature will be added as an entry in the issue tracker. The reason for this is that all features will have a unique identification and description which commits to the repository can be placed against. This is useful for project management reasons as then each feature's stage and completeness can be monitored to ensure the project's success.
		\\\\
		Alternative \glspl{vcs} are available, for example Microsoft's Team Foundation Server or Perforce, however these are limited under free licences or locked to certain development environments. The accessibility and ease of use provided by Git makes it the optimum \gls{vcs} to use for this project. 

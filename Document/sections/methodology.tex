\subsection{Introduction}
	This chapter will give an outline of the methodologies used to complete the work identified in the previous chapters as well as the reasons for using these methods and alternatives which could have been used. The effectiveness of these methods can determine the success of the project so the chosen methods were required to reliable and manageable. A technical approach was identified for the most appropriate way to produce a solution to the problem identified and a project management approach was decided on so that the project could remain on track and meet the required deadline.
\subsection{Literature Review}
	% Purpose
	The purpose of the literature review presented in section \ref{sec:lit_review} of this dissertation is to outline, investigate, and clarify the subject areas in which this project is involved. This aids the reader and author in understanding these subject areas as an explanation as to how the project presents a viable solution to the problem area. To carry out the review a methodology had to be chosen.
	\\\\
	% Types
	There are multiple types of literature review though they do not have the same aims. A traditional or narrative review can be carried out to critique a body of literature and potentially locate inconsistencies \todo{cite}; this type is normally used when the research question is well defined. Alternatively a systematic literature review can be produced; this requires a more rigorous process. Systematic reviews aim to locate studies which are within the same or similar subject areas and use them to answer the questions presented by the research \todo{cite}.
	\\\\
	% Choice
	For this project a loose combination of the two has been used. A systematic approach was applied to examine image analysis uses and techniques to answer how they could be used in the context of mountain bike suspension. A traditional method has been applied to examine the uses of image analysis in sports science to determine if the techniques would transfer to mountain bike suspension and give an idea of how effective it could be. 
	\subsubsection{Source Selection}
		For both methods sources were required to be found which requires a methodical approach to ensure they are suitable. Multiple services are available for viewing academic papers online; for this project the Edinburgh Napier University library and Google Scholar were utilised as between them they have an expansive library to select from and allow for the use of advanced search operators.
		\\\\
		Due to the emerging and more corporate subject of applying technology to mountain bike suspension, there were no academic sources to use; because of this multiple mountain bike websites and blogs have been cited. In a context where the research is more established this would be frowned upon as this type of source tends to be opinionated and unproven though due to the age of this subject this was unavoidable. To counteract this, multiple sources have been cited on the same subject as cross reinforcement.
		\\\\
		To determine if any source is suitable, a methodical and critical approach must be taken when reading the paper or article. Initially a decision can be made before reading by using the publication date to ensure the subject and research is up to date, steps to use research which is the most up to date as possible were taken during this project. Following this, the number of times a paper has been cited elsewhere is provided in online libraries. From this metric it can be determined that if a paper has been cited a higher number of times then it is more suitable and legitimate; this must view should be taken with reservations. Finally when reading a paper the abstract should be read first followed by conclusions and if it is still considered good research, the body can be read. This reduces the amount of reading required as the abstract and conclusions provide a good indication of a suitable paper.
\subsection{Platform}
	As a proof of concept, the final product of this software project could take a variety of forms. Inline with current products on the market an application for Android could be produced which would demonstrate the capabilities of image analysis on a mobile device. Alternatively, the image analysis algorithm can be produced in the Python programming language as a script creating a simpler prototype but clarifying the solution to the problem.
	\subsubsection{OpenCV}
		OpenCV is an open-source computer vision library created for a variety of platforms. Originally released by Intel in 1999\todo{cite} the library was intended as a research project to aid in CPU intensive visual applications. In 2012 the library was taken over by OpenCV.org\todo{cite}, a non-profit organisation who maintain support and the documentation. Though the library is written in C and C++, modules are available for Python, Fortran, and the Android platform.
		\\\\
		This project will use OpenCV as it is widely accepted as the best, free computer vision library available\todo{cite?} and provides functions for the various processes this project will require. Additionally it is well supported and documented which will aid in the development process.
	\subsubsection{Experimentation}
		To further understand what may be required to produce both the Android and Python based approaches and aid in deciding which method to use some basic experiments were carried out to compare the two. For the Android experiments some example applications which are bundled with the OpenCV library were hand copied and run on a mobile device. This allowed their output and functionality to be seen as well as experience in using OpenCV on Android. For the Python based approach, tutorials from the site \url{www.pyimagesearch.com} created by Adrian Rosebrock and \url{www.pythonprogramming.net} creators name unknown were followed and run on a desktop computer.
		\\\\
		Appendix \ref{app:android_experiments} shows the experiments which were carried out on the android platform. It should be noted that, while all applications do not show compilation errors and were adjusted to work with the updated version of Android which the device was using, only two of the four experiments work successfully. Although a stack trace of the errors produced they are not descriptive about the cause of the error. This is due to how OpenCV works on the Android system.
		\\\\
		Alongside the Android \gls{sdk}, Google provide a \gls{ndk} to allow modules which were not written in Java to be run on Android devices. The \gls{ndk} understands various programming languages and applied a Java wrapper around them which is capable of extracting their functionality; this must be applied when using OpenCV on Android as it is written in C++. An artefact of using the \gls{ndk} is that it cannot convert the stack trace produced by errors in the C++ module and after research into this issue it was found that this is difficult to rectify.
		\\\\
		The two working Android experiments do not operate as expected either. Shown in appendix \ref{app:android_experiments} the issues with each can be seen. Although steps were taken to rectify the orientation in the Hello CV experiment and the full-screen issue in both, any fix which was applied was not accepted by the system. Research and debugging of these faults did not produce any conclusions. Appendix \ref{app:python_experiments} shows the experiments which were carried out using Python. These were much more successful than the Android experiments as they are all functional and work as expected.
		\\\\
		Included in appendix \ref{app:android_experiments} and \ref{app:python_experiments} as a comparison between the two methods are the number of files and lines of code required to create the program as well as the relevant source code for each program. There is a clear difference between the two methods. The Android method taking an average of 295 lines of code over 3 files to produce arguably less complex and less functional applications than is produced by Python's 21 lines of code over 1 file.
		\\\\
		Due to the difficulties when using OpenCV on Android and as the operating system is already a proven platform for mobile applications the decision has been made to produce a poof of concept for the image analysis in Python. This will allow the project to maintain focus on the analysis and algorithmic side of the problem as opposed to the portability. The decision also allows for more time to be spent making the program functionally stable which will serve as a better demonstration of the solution.
\subsection{Source Code}
	\subsubsection{Pythonic Coding}
		A metric of software quality is the conciseness and descriptiveness of the source code. The aim of the Python programming language is to complete the same tasks as other object oriented languages but in fewer lines and in a more readable manner\todo{cite}. By removing brackets and instead using indentation to define the boundaries of classes, functions, and conditionals as well as using worded operators, Python creates source code which reads like a list of instructions for people rather than computers. The source code for this project will be written in a pythonic way where it makes sense to do so.
\begin{listing}[ht]
\begin{minted}
[frame=lines,
breaklines,
fontsize=\footnotesize,
linenos]
{python}
long_string = 'This is a very long string'
if 'long' in long_string:
    print 'Match found'
\end{minted}
\caption{An example of Python code}
\label{lst:python_example}
\end{listing}
	\subsubsection{Naming}
		%Descriptive vs Short
		As well as the readability which comes from Python, further efforts will be made to ensure all classes, methods, and variables will be named as descriptively as possible. This makes the system and its algorithms easily understandable from an outside perspective and if any parts of the code need to be revisited at a later date. This is demonstrated in listing \ref{lst:bad_naming_examples}, although the left code looks cleaner, when reading through it is extremely generic and could be part of any system. In contrast the right code is much more descriptive of its function.
\begin{figure}[!h]
	\begin{minipage}{0.5\textwidth}
		\centering
		\begin{minted}
		[frame=lines,
		breaklines,
		fontsize=\footnotesize,
		linenos]
		{java}
public List<int[]> getThem() {
  List<int[]> list1 = new ArrayList<int[]>();
  for (int[] x : theList)
    if (x[0] == 4)
      list1.add(x);
  return list1;
}
		\end{minted}
	\end{minipage}
	\begin{minipage}{0.5\textwidth}
		\centering
		\begin{minted}[frame=lines,
		breaklines,
		fontsize=\footnotesize,
		linenos]
		{java}
public List<int[]> getFlaggedCells() {
  List<int[]> flaggedCells = new ArrayList<int[]>();
  for (int[] cell : gameBoard)
    if (cell[STATUS_VALUE] == FLAGGED)
      flaggedCells.add(cell);
  return flaggedCells;
}
		\end{minted}
	\end{minipage}
	\captionof{listing}{Examples of bad naming (left) and proper naming (right) taken from Clean Code \citep{martin2009clean}}
	\label{lst:bad_naming_examples}
\end{figure}
	\subsubsection{Object Orientation}
		Object orientation is the use of classes which contain their specific variables and methods to protect functionality from other parts of a system. An object oriented approach will be taken when creating the system to ensure its functions and data is protected should it ever be use as a module elsewhere. The use of private variables and methods with one or two public methods provides an API with which other developers can use the system.
		\\\\
		Python does not provide the private and public keywords as found in the Java or C++ languages. It instead identifies methods and variables prefixed by one or two underscores as protected. This means these items will be hidden from view when the system is used though are accessible should the developer require them. The creators of Python chose this method as they enforce responsibility over restriction.
\subsection{Testing}
	To verify the functionality and quality of a software application it must be tested. This can be carried out at a variety of levels from testing a single class method to an entire system and by knowing how an application works in white-box testing or being unaware of the functionality in black-box testing.
	\subsubsection{Unit Testing}
		Unit testing is carried out on individual units of source code to ensure they are functioning correctly. Normally carried out in the scope of an individual class or module these unit tests are typically written by the same software developer who produced the class itself. A single unit test comprises of a set-up process where data and objects are initialised, the test itself including an assertion on a variable determining pass or fail, and a tear-down process where any changes the test has made are cleared up.
		\\\\
		As unit tests are simple in structure and quick to run then they are commonly executed when changes are made to the source code. This verifies that the changes made have not broken other sections of code and can be committed into the master branch successfully. If a test failure does occur then each unit test should be concise and descriptive enough to aid the debugging process by indicating what has failed and where.
		\\\\
		A metric of quality for unit testing is code coverage. To confirm that a class is of good quality and functional, every possible path through the code must be tested. Testers should aim for 100\% coverage of a module meaning every possible piece of functionality has an associated test, a coverage of 85\% for example means 15\% of a module is not under test and could cause undetected problems if any changes are made. Many IDEs or testing suites provide coverage checking functionality to make this process simple.
		\paragraph{Python Unit Testing}
			Due to the package based and open-sourced nature of the Python programming languages there are many implementations of unit testing frameworks to choose from. Each with their own advantages and disadvantages, there is common argument over which framework is the best to use. A good unit testing framework should provide the necessary functions to create simple unit tests for each path in the source code and be substantial enough to ensure the class or module is correctly tested. 
	\subsubsection{Project Testing Scope}
		Due to the small size of the application which will be produced in this project, testing will be limited to unit testing as there is no integration to be carried out. This small size also means that 100\% code coverage will be simple to achieve. From this, stability and functionality of the application can be verified improving the quality of the overall product.
		\\\\
		Once the basic functionality has been added to the application, unit tests will be created covering all source code produced including normal operation and sections which can potentially produce errors. This means any changes made thereafter can be tested to see if they have affected functionality. Any new functionality will have tests added to the testing suite to keep the 100\% coverage.
		\\\\
		The Python unit testing framework chosen is unittest2. Unittest is the default framework used by the PyCharm IDE though this has been updated since the inception of Python3. Unittest2 provides a backport of this functionality for use in Python2 meaning extra testing functions are provided to allow more suitable tests to be created. PyCharm also provides the ability to run tests with code coverage producing a clear indication of which sections of source code are covered and which are not; this will be greatly beneficial in the aim for full coverage.
\subsection{Project Management}
	\subsubsection{Agile Development}
		\todo{Remove statements saying this will be used}
		Introduced in 2001 with the writing of the Agile manifesto \citep{beck2001manifesto}, agile development methodologies focus on high quality software products over the rigorous design based structure of traditional waterfall methods. By utilising various techniques such as stand-up meetings, a strong customer focus, and sprint cycles, agile has become widely adopted in industry and has been proven to produce successful projects \citep{state_of_agile_2015}.
		\\\\
		There are multiple agile methodologies (XP, Scrum, DSDM) all with their own principles and techniques however it is commonplace for a company to create their own development method which picks items from each to tailor to their needs\todo{cite}. As this is a solo project with no customer then one set methodology will not be used, instead a variety of methods which will aid the management of the project and increase the efficiency of the development process have been selected. These will be outlined in the following sections.
		\paragraph{Requirements Analysis}
			Used in some form by the majority of agile methodologies \todo{cite}, requirements analysis or requirements engineering is a variety of processes used to create the conditions that a project must meet. These requirements take into account stakeholders, users, and the development team or teams. Each requirement should be documented, actionable, measurable, testable, and traceable to aid in its understanding and completion\todo{cite}.
			\\\\
			The techniques used to produce requirements include stakeholder identification and interviews which are used to identify what the stakeholders of the project require upon completion. These interviews may be followed by Joint Requirements Development Sessions which bring stakeholders together to further discuss the requirements of the project. A more traditional approach is to produce a contract-style requirements list though these are commonly extensive and incomplete due to their production without collaboration. The most used requirements analysis technique in agile development are use cases and user stories. These are either written or diagrammatic descriptions of how the system will interact with users or other systems and provide an indication of what will be required from the product.
			\\\\
			Due to the prototype nature of the application being produced and as the project has no defined stakeholders the only requirements analysis technique which will be used is production of use cases. These will aid in identifying how the application will be operated by users and what the project will need to produce. These requirements will then be prioritised using the MoSCoW format and inserted into a tracking system for the development process; these are described in the following sections.
		\paragraph{MoSCoW}
			First used in the DSDM agile framework\todo{cite}, MoSCoW analysis or prioritisation is the process of taking the requirements of a software product and placing them into one of four categories; must, should, could, and won't have. These deliverables may be prioritised for the entire project or for individual sprint cycles depending on the size and manageability of the project.
			\\\\
			MoSCoW was created to provide customers a better understanding of software requirements. Opposed to using high, medium, and low priorities MoSCoW is more descriptive of what the prioritisation means for the software project. From a development point of view the prioritisation process allows developers to focus on the core requirements of the project first, creating a viable software solution early on with extra features being added later if the resources are available.
			\\\\
			This project will use MoSCoW to prioritise the requirements identified using the requirements analysis technique allowing the development process to complete the goals in the correct order. This will ensure that the core aspects of the solution are implemented first creating a successful project early and allowing it to improve as time allows.
		\paragraph{Sprint Cycles}
			Used by Scrum development teams\todo{cite}, a sprint is a timeboxed effort of work scheduled to take between one week and one month. At the start of a sprint goals are chosen from the project requirements which are to be completed by the end of the sprint. When a sprint is complete a retrospective is carried out by the development team to discuss what went well, what didn't, and how this can be rectified for the next sprint.
			\\\\
			This project will use sprints in the same manner as an agile development team as this will allow easier completion of requirements and management of the development process. Using sprints will ensure the time allotted for development is used effectively which will lead to a higher quality product at the end of the project.
			\\\\
			As mentioned previously, MoSCoW prioritised requirements will be selected at the start of each sprint cycle to set objectives the work which will be carried out. Each cycle will aim to complete all of the must have requirements and the majority of should and could haves. At the end of each cycle a retrospective will be carried out which will re-prioritise the uncompleted requirements based on how successful the sprint cycle was. This means requirements may be promoted or demoted respectively.
	\subsubsection{Organic Development}
		% Describe
			An alternative method to agile development could be to take a more organic approach to the development process. This method would take the step by step ethic of agile and reduce the effort applied for analysis, documentation, and structuring of the working schedule.
			\\\\
			% Identify basic steps, start developing, adapt as needed
			To initiate the development process without analysis of requirements would be to begin producing the basic steps or functionality which the system will go through from an vision of the end product. Once started, development would take a natural direction identified by what is needed for the system or to solve any issues encountered.
			\\\\
			% Dev diaries
			To document the development process, a log of any work completed would be kept per day whenever development is carried out. This would allow for referral at later dates and allow the project to remain on track by identifying when features are completed.
			\\\\
		% Reasons
			% Reduces over-management
			There are advantages and disadvantages to using this method over an agile process. Due to the size of the software being produced the amount of management work required of an agile process could mean the development is over managed
			% Allows easier problem solving
			% Could cause project to run out of time but is more dynamic
		
	%%%%%%%%%%%%%%%%%%%%%%%%% SEGWAY %%%%%%%%%%%%%%%%%%%%%%%%%%%%%%%%%%%%
	\subsubsection{Version Control}
		Large software projects produce multiple files of code and documentation which are vital and must be kept safe. Were the files to be lost then the project would be delayed or drawn to a close as the time and resources may not be available to recreate the lost data. To combat this any data relating to the project must be backed up, preferably on a cloud based system, to avoid loss and allow the project to continue should anything happen to the local copy of the data.
		\\\\
		For this project the Git \gls{vcs} will be used. Git provides use of a cloud hosted repository to store any files relating to a project and allows for work to be carried out locally by cloning the repository on a computer. However \glspl{vcs} also enable the management of the previous versions of files including information such as the individual changes made, when those changes were made, and who made the changes. This is a powerful tool as it means the various sections of the project can be experimented on without the risk of damaging the project; if a change is unsuccessful then the repository can be reverted to a functional point.
		\\\\
		The web service used to host the repository will be github.com. This site was chosen as it provides unlimited free repositories as well as simple repository management tools. The website also provides issue tracking functionality; though they are not issues, each feature will be added as an entry in the issue tracker. The reason for this is that all features will have a unique identification and description which commits to the repository can be placed against. This is useful for project management reasons as then each feature's stage and completeness can be monitored to ensure the project's success.
		\\\\
		Alternative \glspl{vcs} are available, for example Microsoft's Team Foundation Server or Perforce, however these are limited under free licences or locked to certain development environments. The accessibility and ease of use provided by Git makes it the optimum \gls{vcs} to use for this project. 
	\subsubsection{Milestones}
		Milestones are used in project management to mark significant events or points within a project's timeline. This allows a further breakdown of the project as milestones can be used as supplementary deadlines. The use of milestones allows the project manager, management team, or development team to keep track of the project's status and priorities at any given time.
		\\\\
		This project has and will use milestones for the same reason. Current examples include the week nine review session and deadline for the completion of this document. As more milestones are identified through breakdown of the development process they will be documented and acted upon.
	\subsubsection{Threshold}
		Tasks using more than their allotted time in a common cause of projects using more of their resources. To ensure all tasks can be completed within the allotted time for this project, each task will be allocated a threshold. Adding a threshold is a common technique in project management and provides extra time should anything unexpected occur which impacts the project.
	\subsubsection{Gantt Chart}
		A Gantt chart is a method of plotting a project's schedule which was invented by Henry Gantt in the 1910s. By breaking down a project into tasks these can then be added to the chart as a bar. Commonly starting at a project's inception date and ending with it's completion date, tasks are allocated an estimated duration for their completion and placed within the timeline. These tasks can then be allocated dependencies to indicate their prerequisites.
		\\\\
		This project will make full use of the Gantt chart technique by reducing the project into multiple stages for both writing and development. Previously mentioned milestones will also be added for the known fixed points. Each sprint cycle for the development process will be indicated with further notes on the tasks to be carried out. To create the chart Microsoft Project 2016 will be used.
\subsection{Evaluation}
	To ensure that the application produced is extensively evaluated, multiple methods will be used. Each of the different methods will evaluate a single aspect of the application from varying points of view.
	\subsubsection{Validation}
		The is a measure of how well a design or solution is appropriate for its purpose and performs as expected. It involves checking all required functionality is present, each output is correct and as expected, and the solution performs as expected.
		\\\\
		In this project, validation will be provided through the implemented unit tests. Ensuring the application has near 100\% coverage and all tests are passing then it can be confirmed that the application is operating as expected.
	\subsubsection{Reliability and Accuracy}
		Assessing the reliability and accuracy of the application proves that the results produced are consistent and correct; to assess these an uncertainty metric will be produced. When taking multiple measurements there is typically a "true" value which is the actual measurement that falls within the range of produced measurements. Uncertainty is a prediction of how close any produced measurement will be to this "true" value, expressed as $\overline{x} \pm U$. The method for producing uncertainty is as follows:
		\begin{enumerate}
			\item Produce a suitable number of repeat measurements $\{x_0 ... x_n\}$
			\item Calculate the average value of these measurements $\overline{x}=\frac{\sum\{x_0...x_n\}}{n}$
			\item Find the differences between the measurements and the average $\{d_0...d_n\} = \{(x_0-\overline{x})...(x_n-\overline{x})\}$
			\item Calculate the average of these differences squared $\overline{d_s} = \frac{\sum\{{d_0}^2...{d_n}^2\}}{n}$
			\item Produce the uncertainty or standard deviation of these results which is the square root of the average differences $U= \sqrt{\overline{d_s}}$
		\end{enumerate}
		This process will be carried out a number of times altering different variances possible depending on what the application is capable of once produced. These variances will be outlined once the evaluation has been completed.
	\subsubsection{Comparison to Alternatives}
		By comparing the process of using the application with other methods of producing a sag setting it can be evaluated on how simple the application is to use and whether it is more effective than other methods. For this comparison a trial and error process will be used which would be a common practice for a beginner or intermediate rider setting their suspension for the first time. This process involves manually calculating the desired sag measurement, pressurising the shock to the manufacturers recommended pressure, loading the suspension with the rider's weight, and measuring. If the manufacturers setting is incorrect, pressure is then increased or removed by set increments. Comparison to this process will indicate that the application presents a benefit over the manual method.
	\subsubsection{Professional Opinion}
		The final method of evaluation will be to seek the opinion of the application from professionals within the mountain bike and cycling industry. This will provide a further indication of how well the application achieves its goal, how the application would be accepted were it to be released as a product, and may present areas for future work which may not have been seen without outside consultation.
		\\\\
		To carry out professional consultation the Mountain Bike Research Centre of Scotland\footnote{http://www.napier.ac.uk/about-us/our-schools/school-of-applied-sciences/research/mountain-bike-centre-of-scotland} will be contacted, once the application can be demonstrated, to arrange a meeting with an individual who they deem suitable. General evaluative questions will be posed though the meeting will take a semi-formal approach. Additional to this the application will be demonstrated to staff at a local bike shop to also gain their opinion, this will take the same approach as the meeting.
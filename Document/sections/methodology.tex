\subsection{Introduction}
	This chapter will give an outline of the methodologies used to complete the work identified in the previous chapters as well as the reasons for using these methods and alternatives which could have been used. The effectiveness of these methods can determine the success of the project so the chosen methods were required to reliable and manageable. A technical approach was identified for the most appropriate way to produce a solution to the problem identified and a project management approach was decided on so that the project could remain on track and meet the required deadline.
\subsection{Platform}
	\subsubsection{Experimentation}
\subsection{Project Management}
	\subsubsection{MoSCoW}
	\subsubsection{Requirements Analysis}
	\subsubsection{Version Control}
		Large software projects produce multiple files of code and documentation which are vital and must be kept safe. Were the files to be lost then the project would be delayed or drawn to a close as the time and resources may not be available to recreate the lost data. To combat this any data relating to the project must be backed up, preferably on a cloud based system, to avoid loss and allow the project to continue should anything happen to the local copy of the data.
		\\\\
		For this project the Git \gls{vcs} will be used. Git provides free use of a cloud based repository to store any files relating to a project and allows for work to be carried out locally by cloning the repository on a computer. However \glspl{vcs} also enable the management of the previous versions of files including information such as the individual changes made, when those changes were made, and who made the changes. This is a powerful tool as it means the various sections of the project can be experimented on without the risk of damaging the project; if a change is unsuccessful then the repository can be reverted to a functional point.
		\\\\
		
	\subsubsection{Agile}

\subsection{Introduction}
	This chapter will give an outline of the methodologies used to complete the work identified in the previous chapters as well as the reasons for using these methods and alternatives which could have been used. The effectiveness of these methods can determine the success of the project so the chosen methods were required to reliable and manageable. A technical approach was identified for the most appropriate way to produce a solution to the problem identified and a project management approach was decided on so that the project could remain on track and meet the required deadline.
\subsection{Platform}
	\subsubsection{Experimentation}
\subsection{Project Management}
	\subsubsection{Agile Development}
		Introduced in 2001 with the writing of the Agile manifesto \citep{beck2001manifesto}, agile development methodologies focus on high quality software products over the rigorous design based structure of traditional methods. By utilising various techniques such as stand-up meetings, a strong customer focus, and sprint cycles, agile has become widely adopted in industry and has been proven to produce successful projects \citep{state_of_agile_2015}.
		\\\\
		There are multiple agile methodologies (XP, Scrum, DSDM) all with their own principles and techniques however it is commonplace for a company to create their own development method which picks items from each to tailor to their needs. As this is a solo project with no customer then one set methodology will not be used, instead a variety of methods which will aid the management of the project and increase the efficiency of the development process have been selected. These will be outlined in the following sections.
		\paragraph{Requirements Analysis}
		\paragraph{MoSCoW}
			MoSCoW analysis or prioritisation is the process of taking the requirements of a software product and placing them into one of four categories, must, should, could, and won't have. These deliverables may be prioritised for the entire project or for individual sprint cycles depending on the size and manageability of the project.
			\\\\
			By defining the priority of the project's features the core functionalities can be completed before others 
		\paragraph{Sprint Cycles}
	\subsubsection{Version Control}
		Large software projects produce multiple files of code and documentation which are vital and must be kept safe. Were the files to be lost then the project would be delayed or drawn to a close as the time and resources may not be available to recreate the lost data. To combat this any data relating to the project must be backed up, preferably on a cloud based system, to avoid loss and allow the project to continue should anything happen to the local copy of the data.
		\\\\
		For this project the Git \gls{vcs} will be used. Git provides free use of a cloud based repository to store any files relating to a project and allows for work to be carried out locally by cloning the repository on a computer. However \glspl{vcs} also enable the management of the previous versions of files including information such as the individual changes made, when those changes were made, and who made the changes. This is a powerful tool as it means the various sections of the project can be experimented on without the risk of damaging the project; if a change is unsuccessful then the repository can be reverted to a functional point.
		\\\\
		The web service used to host the repository will be github.com. This site was chosen as it provides unlimited free repositories as well as simple repository management tools. The website also provides issue tracking functionality; though they are not issues, each feature will be added as an entry in the issue tracker. The reason for this is that all features will have a unique identification and description which commits to the repository can be placed against. This is useful for project management reasons as then each feature's stage and completeness can be monitored to ensure the project's success.
		\\\\
		Alternative \glspl{vcs} are available, for example Microsoft's Team Foundation Server or Perforce, however these are limited under free licences or locked to certain development environments. The accessibility and ease of use provided by Git makes it the optimum \gls{vcs} to use for this project. 

	\subsection{Mountain Bike Suspension Concepts}
		The purpose of suspension on a mountain bike is to absorb the  energy created from riding over features such as bumps and rough terrain encountered along  a trail,  improving  comfort for the rider and allowing them to go faster by maintaining better contact between the tires and the ground. This requires the use of a spring and damper, collectively known as a shock absorber, which allows the wheel to move away from the feature on contact  and make a controlled return once it has been passed.
	\subsubsection{Travel and Stroke}
		\Gls{travel} is the distance which the bike’s fork or frame allow the wheel to move in an upward direction while \gls{stroke} is the distance that the shock absorber can compress before it bottoms out. \Gls{travel} is measured in millimetres or inches and can range from 80mm to 210mm or 4in to 9in. Bikes designed for different disciplines require differing amounts of travel with those designed for cross-country riding typically requiring less travel than those designed for more aggressive disciplines such as downhill racing having more. See Table \ref{tab:travel} for typical specifications.
		\begin{table}[h!]
		\centering
		\caption{Table of common suspension \glspl{travel} and intended disciplines}
		\label{tab:travel}
		\begin{tabular}{|c|cccc|}
			\hline
			Travel (mm)&Cross Country&Trail&Enduro&Downhill\\
			\hline
			80&\cellcolor[gray]{0.5}&&&
			\\
			100&\cellcolor[gray]{0.5}&&&
			\\
			120&\cellcolor[gray]{0.5}&\cellcolor[gray]{0.5}&&
			\\
			140&&\cellcolor[gray]{0.5}&\cellcolor[gray]{0.5}&
			\\
			160&&&\cellcolor[gray]{0.5}&
			\\
			180&&&\cellcolor[gray]{0.5}&\cellcolor[gray]{0.5}
			\\
			+200&&&&\cellcolor[gray]{0.5}\\
			\hline
		\end{tabular}
	\end{table}
	\begin{figure}[h!]
		\centering
		\includegraphics[width=12cm]{../images/reignschpath.PNG}
		\caption[Diagram showing travel and stroke on a full suspension bike]{Diagram showing 
			travel and stroke on a full suspension bike\footnotemark}
		\label{fig:travelvsstroke}
	\end{figure}
	\subsubsection{Front Suspension}
		Front suspension commonly employs a linear telescoping shock absorber, known as a \gls{fork} due to it's dual sided construction. On nearly all suspension \glspl{fork} the \gls{stroke} is 1:1 with the potential travel of the wheel. Front suspension is found on 
		all \gls{fs} and \gls{ht} bikes.
	\subsubsection{Rear Suspension}
		
		Rear suspension uses a shock absorber that is much shorter than a fork so  it cannot operate on a 1:1 ratio and still allow the desired travel. Full suspension frames incorporate one or more pivot points and linkages which allow the wheel to move and act as multipliers for the suspension. Rear ratios are expressed as n:1 where n is the average distance the rear wheel moves for every 1mm the shock compresses throughout its stroke; the leverage ratio changing constantly across the compression cycle.
		\\\\
		The difference between front and rear stroke and travel can be seen in Figure \ref{fig:travelvsstroke} noting the separation of rear wheel travel from the stroke of the shock absorber. Although manufacturers design their rear suspension differently, the rear wheel always  rotates around the main pivot, (or in some cases a virtual pivot), as opposed to moving linearly as front forks do. As a result, the frame  behaves differently through its travel, depending on the number and location of pivot points and the type of shock that it is being used. 
		\\\\
		Because of this the average ratio is normally dismissed in favour of a leverage curve that plots the ratio n:1 throughout the compression cycle. Figure \ref{fig:3_bike_lev_ratio} shows the leverage curves of three modern suspension designs. Each of these designs has between 150mm and 170mm of travel and uses the 27.5 inch wheel size. However, it is evident that varying the location of pivot points produces suspension with  drastically different characteristics.
		\todo{Figure of 3 bikes overlayed}
		\begin{figure}[h!]
			\centering
			\includegraphics[width=10cm]{../images/3_bike_lev_ratio.jpg}
			\caption{Leverage curves of three modern suspension designs}
			\label{fig:3_bike_lev_ratio}
		\end{figure}
		\\\\
		The \gls{vpp} design of the Giant Reign (shown blue) has an initial falling rate, meaning the \gls{shock} can be compressed easily, but slows down and even rises slightly towards 
		the end of its travel; this frame is indicated in figure \ref{fig:travelvsstroke}. This 
		means the suspension will feel soft most of the time but feel stiffer on large 
		compressions. This is emphasised by the \gls{horst} system of the Lapierre Spicy (shown 
		magenta) which has a large rise at the end of its travel.
		\\\\
		In contrast, the curve of the \gls{singlepiv} Empire MX-6 Evo (shown green) is considered linear. This is due to the MX-6 having only one pivot and swinging arm, as opposed to multiple pivots and linkages of the \gls{vpp} and \gls{horst} designs, so there is an almost direct input from the rear wheel to the \gls{shock}.
		\\\\
		For this project the bike used for development and testing of the application will be a 2015 Giant Reign, shown on figure \ref{fig:3_bike_lev_ratio} in blue, as there will be constant access to it during the project. The frame uses Giant's Maestro\texttrademark  suspension system which is a variation of \gls{vpp}. Like all \gls{vpp} systems Maestro uses two links, an upper and lower, to create a virtual main pivot point, however unlike other \gls{vpp} systems, indicated by the red circle in Figure \ref{fig:maestro}, Maestro creates its virtual pivot as close to the rear of the frame as possible.
		\begin{figure}[h!]
			\centering
			\includegraphics[width=12cm]{../images/reignsch.PNG}
			\caption{Maestro suspension}
			\label{fig:maestro}
		\end{figure}
		\\\\
		Despite the complexities relating to rear suspension designs, the average rider and even individuals in the industry are unlikely to need this knowledge. Leverage curves are predominantly used by designers in the research and development of new frames to determine the characteristics of how they will use the suspension available before the new bike is released to the market\todo{cite}. Though an understanding of these characteristics will allow the individual rider to tune the settings described in the following sections themselves, the majority will not notice the difference between a single pivot and \gls{vpp} design, opting for a more simple "set and forget" approach to their suspension.
		\\\\
		In the context of this project and the intended user for the application, this begs the question of how much information should be given to the user? Modern human computer interaction principles aim toward providing information to the user which is relevant and necessary in context\todo{cite}. Presenting the user with a leverage curve which could require extensive explanation to understand within a mobile application could prove detrimental to the user experience\todo{cite?}. For an application which is intended to remove the difficulties of suspension setup and allow for quick and easy production of a basic setup, providing a single sag setting would be preferable over a multitude of information. 
	\subsubsection{Sag}
		\Gls{sag} is the amount that the suspension sits into its travel when the rider is in their neutral position, it is calculated using the rider's weight, available \gls{travel}, and intended riding style. \Gls{sag} is required so the suspension has travel available to drop into holes as well as soak up bumps. 
		\\\\
		To adjust \gls{sag}, the stiffness of the spring must be adjusted as required. This is done by changing the air pressure when using an air spring or replacing the coil and adjusting the spring pre-load on traditional coil \glspl{shock}. Depending on discipline and the amount of \gls{travel} the bike has, \gls{sag} can vary between 15\% and 40\% of the available travel though is commonly set between 25\% and 35\% for the average rider. 15\% and 40\% are reserved for competitive situations. 
	\subsubsection{Damping}
		Suspension damping is carried out by forcing oil within the shock absorber through an arrangement of holes in the absorber's damping circuit. Reducing the size or number of holes making the travel of oil through the circuit slower and therefore increases the damping effect making compression or rebound slower.
	\paragraph{Compression Damping} 
		This is applied while the shock absorber is being compressed. More damping forces the wheel to remain in contact with the ground which makes the suspension feel stiffer. Too much compression damping can make the suspension too stiff so it does not soften bumps or rough sections correctly. Too little can cause the suspension to "blow through" its travel prematurely potentially leaving none when it would be required.
	\paragraph{Rebound Damping}
		This is used to control the speed at which the shock absorber extends once it has been compressed. An optimal setting will allow the suspension to track the ground, returning after a bump as well as dropping into any holes. Too much \gls{rebounddamping} causes the suspension to return slowly and sometimes pack down meaning the absorber gradually runs out of travel. Too little can cause the suspension to buck the rider and lead to an accident.
	\paragraph{High and Low Speed Damping} 
		Depending on the manufacturer and model of the shock absorber, the unit can include up to two adjustable speeds for each damping circuit making four adjustable damping settings in total. High speed adjustments are used in high impact situations such as large jumps or drops, compression tends to be set softer to remove impact and rebound slower so rider has time to recover and the bike is not made unstable.
		\\\\
		Low speed adjustments are used against small movements such as rider weight shifts or long, slow compressions. Optimally compression is set stiffer as this type of feature can use a lot of travel and \gls{rebounddamping} set faster to deal with multiple features in quick succession.
	\subsubsection{Optimal Setup}
		Although setups will vary between rider, suspension system, and discipline there are some key aspects which all riders should aim to achieve. Sag should be set to an appropriate measurement by adjusting the air pressure on air \glspl{shock} or spring rating on coil \glspl{shock}. Compression damping should feel soft and soak up bumps efficiently without excessive bottoming out. Rebound damping should be set to return as fast as possible without bucking the rider, this is normally in the middle of the two extremes of setting with a slight bias to the fast option.
		\\\\
		Attaining this optimal setup can be difficult for beginner and intermediate riders; due to their lack of experience and provided information they may not know how different frames react while being ridden and have not dealt with in depth suspension setup. Knowing which measurements to make and calculations required to produce a sag setting are commonly unknown to this level of rider without prior training or investigation.
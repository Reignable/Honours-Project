\documentclass[a4paper, 12pt, hidelinks]{article}

\title{Honours Project Dissertation}
\author{Joe Barrett}

\usepackage[natbibapa]{apacite}
\bibliographystyle{apacite}
\usepackage{blindtext}
\usepackage{eurosym}
\usepackage{fancyhdr}
\usepackage{fontspec}
\usepackage[toc,acronym]{glossaries}
\usepackage{hyperref}
\usepackage{url}

\usepackage[ddmmyyyy]{datetime}
\usepackage[margin=1in]{geometry}
\usepackage[none]{hyphenat}

\setmainfont{Arial}

\pagestyle{fancy}
\renewcommand{\headrulewidth}{0pt}

\lhead{Joe Barrett - 40117680}
\rhead{SOC10101 Honours Project}
\cfoot{\thepage}

\makeglossary
\loadglsentries{glossary}

\begin{document}
	\begin{titlepage}
	\begin{center}
		
		\vspace*{3cm}
		{\LARGE Calculation of Mountain Bike Suspension Settings through Image Analysis}
		
		\vspace{3cm}
		{\large Joe Barrett - 40117680}

		\vfill
		Submitted in partial fulfilment of the requirements of Edinburgh Napier University for the Degree of BEng (Hons) Software Engineering
		
		\vspace{1cm}
		School of Computing
		
		\vspace{1cm}
		\today
	\end{center}
\end{titlepage}
	\begin{abstract}
		\blindtext
	\end{abstract}
	\newpage
	\tableofcontents
	\newpage
	\listoftables
	\newpage
	\listoffigures
	\newpage
	\section{Introduction}
	\subsection{Context}
	A survey carried out by the International Mountain Bike Association shows the average price of mountain bikes owned in Europe to be \euro2546 (\pounds2206) \citep{imbasurv}. Starting at around \pounds1000 \citep{giantstance}, enthusiast level mountain bikes can be purchased with suspension for both the front and rear wheels, known as full suspension bikes. Even at this comparably low cost, the suspension units have multiple adjustments available to tune and personalize how they operate.
	\\\\
	To ensure the \gls{fork} and \gls{shock} function correctly they must be set up for the rider's weight and intended use of the bike. As this is considered a specialist area, many entry and mid level riders will lack the knowledge of this process meaning the rider could use the bike without the suspension set up correctly.
	\\\\
	It has been proven that using a \gls{fs} over a \gls{ht} offers a performance advantage to the rider \citep{fullsusperf}. However if the suspension fork and/or shock have not been set up, it can be detrimental to the rider's performance and potentially lead to injury. For example, if a shock has too little \gls{rebounddamping} set and the rider goes off a jump, the excessive speed at which the rear of the bike extends can create forwards rotation, causing the rider to go over the handlebars of the bike.
	\\\\
	Additionally, an incorrect suspension setup can cause excessive wear and tear on the bike's frame and components. Suspension which is set too soft can allow for bottoming out which expends excess forces into the frame and potentially cracks the frame's structure. Suspension set too hard forces energy which it would normally soak up to be forced into the wheels and tires causing denting and warping of the wheel rims.
	\\\\
	Many bicycle retailers will set up the suspension on a newly purchased mountain bike for the customer on delivery. Most of the time this will be enough to avoid incident but due to the extra weight of the equipment riders use, i.e. helmet, hydration pack, body armor, this setup is regularly inaccurate. Furthermore, with some manufacturers choosing direct sales over local retailers \citep{roseonline, ytonline}, this setup can be circumnavigated altogether.
	\newpage
	\section{Literature Review}
	\newpage
	\section{Approach}
	\newpage	
	\section{Results}
	\newpage
	\section{Critical Evaluation}
	\newpage
	\section{Conclusion}
	\newpage
	\bibliography{bibliography}	
	\newpage
	\printacronyms
	\printglossary[type=main]
	
\end{document}